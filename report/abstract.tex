\selectlanguage{english}

\begin{abstract}

    Although the development of the \textbf{i-vector}-based probabilistic linear
    discriminant analysis (PLDA) systems led to promising results in speaker
    recognition, the impact of \textbf{domain mismatch} when the system training
    data and the evaluation data are collected from different sources remains a
    challenge. Johns Hopkins University (JHU) 2013 speaker recognition workshop,
    for which a domain adaptation challenge (DAC13) corpus was created, focused
    on finding solutions to address this problem. \\

    This research report lays out the state-of-the-art techniques used for
    domain mismatch compensation ; such as a combination of various
    \textbf{whitening} transforms, and the use of a \textbf{dataset-invariant}
    covariance normalization to obtain domain-invariant representations of PLDA
    training data. Those techniques are evaluated on the DAC13 corpus and
    compared.

\end{abstract}

\selectlanguage{french}

\begin{abstract}

    Bien que le développement des systèmes d'analyse discriminante linéaire
    probabiliste (PLDA) basés sur les i-vecteurs a donné lieu à des résultats
    prometteurs en reconnaissance du locuteur, l'impact du \emph{domain
    mismatch} lorsque les données d'entraînement du système et les données
    d'évaluation proviennent de sources différentes reste un défi. Le workshop
    de reconnaissance du locuteur de 2013 de l'Université Johns Hopkins (JHU),
    pour lequel un corpus d'adaptation du domaine (DAC13) a été créé, a
    travaillé à trouver des solutions pour résoudre ce problème. \\

    Ce rapport de recherche présente les techniques de pointe utilisées pour la
    compensation du \emph{domain mismatch} ; comme une combinaison de plusieurs
    transformées de blanchiment, et la normalisation de la covariance
    indépendante du jeu de données pour obtenir des représentations des données
    d'entraînement de la PLDA invariantes par rapport au domaine. Ces techniques
    sont évaluées sur le corpus DAC13 et comparées.

\end{abstract}

\selectlanguage{english}
